\documentclass[12pt]{article}
\usepackage{graphicx}
  
%
% Title.
\title{EE230: Experiment 3\\
Instrumentation amplifier and load cell sensor}

% Author
\author{Name of student, Roll. no.}

% begin the document.
\begin{document}

% make a title page.
\maketitle

\section{Overview of the experiment}

\subsection{Aim of the experiment}

In your own words, describe the aim of the experiment.

\subsection{Methods}

In your own words, describe how you set out to realize the goal of the experiment. Only 1 paragraph of a brief overview of your approach is expected here. Do not list your observations here.

\section{Design of load cell}

Based on slide 4 of handout, can you make a rough drawing of how you think the 4 strain gauges must be mounted on the cantilever to realize the load cell that is present in the weighing scale you used in lab? (Scan of a hand-drawn sketch is also acceptable as a figure, brownie points for making good-looking sketch in inkscape, powerpoint, MS Visio etc.)

Write a short explanation (5-6 sentences are more than sufficient) for how your guesstimate of the construction works.

For the wheatstone bridge, derive the relation between bridge voltage and change in resistance (handout, slide 8, bullet point 4)

\section{Simulation results}

In your own words, describe how you set up the simulation, and document your observations. Also copy-paste your own simulation code (netlist) here.

What challenges did you face in simulation? Were there any discrepancies compared to what you expect from KCL-KVL based analysis?

\section{Experimental results}

\subsection{Part 1: Three op-amp implementation}

Draw the circuit diagrams of the circuit you used, with the resistor values you used marked clearly. Only diagrams made in xcircuit will be accepted. Copy-pasting from handout will be counted as plagiarism. If you are smart, you will start drawing the most complicated circuit first, and then modify it accordingly for the other diagrams.

Mention your observations, following instructions in handout. What challenges did you face in performing this part of the experiment?

Include your picture of the DSO screen for the measurement of part 1. Comment on the waveform  you see - were you expecting such a response? (you had to set the DSO in AC coupled mode for this part. If you used DC coupling, borrow somebody else's picture with due credit given in your report).

\subsection{Optional - Part 2: Using PCB 3-op-amp implementation}

Write this sub-section only if you performed Part 2. Part 2 is optional. Many of you may not write this sub-section; that is fine.

Mention your observations. What challenges did you face in performing this part of the experiment? How do they compare to part 1?

\subsection{Part 3: Three op-amp implementation}

Draw the circuit diagrams of the circuit you used, with the resistor values you used marked clearly. Only diagrams made in xcircuit will be accepted. Copy-pasting from handout will be counted as plagiarism. If you are smart, you will start drawing the most complicated circuit first, and then modify it accordingly for the other diagrams.

Mention your observations, following instructions in handout. What challenges did you face in performing this part of the experiment? Compare your observations to part 1.

Include your picture of the DSO screen for the measurement of part 3. Comment on the waveform  you see - were you expecting such a response? (you had to set the DSO in AC coupled mode for this part. If you used DC coupling, borrow somebody else's picture with due credit given in your report).

\section{Questions for reflection}

1. Explain the waveform you see on DSO. What could be the reason for this (this is called a hypothesis - where you are making a well educated guess)? How will you verify your hypothesis (suggest experiments that will either prove or disprove your theories)\\
Ans. Enter answer here. Do not copy!! Beware of turnitin (the tool that we will use to catch copy-cats shown in lab on Tuesday).
\\\\
2. Can you implement an instrumentation amplifier with 2 op-amps? Read up some reputed references on the internet or some reputed textbook, and explain in your own words. (Turnitin catches verbatim copying cases, from each others' reports as well as from the internet!)\\
Ans. Enter answer here.
\\\\
3. What would happen if you did not connect capacitor C1 in the circuit in part 3?\\
Ans. Enter answer here.
\\\\
4. Will you copy or cheat in this lab and other courses, now that you are aware of Turnitin? Explain your answer.\\
Ans. Enter answer here.

\end{document}
