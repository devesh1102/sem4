\documentclass[12pt]{article}
\usepackage{graphicx}
  
%
% Title.
\title{EE230: Experiment 2\\
Non-idealities in Op-amps}

% Author
\author{Name of student, Roll. no.}

% begin the document.
\begin{document}

% make a title page.
\maketitle

\section{Overview of the experiment}

\subsection{Aim of the experiment}

In your own words, describe the aim of the experiment.

\subsection{Methods}

In your own words, describe how you set out to realize the goal of the experiment. Only 1 paragraph of a brief overview of your approach is expected here. Do not list your observations here.

\section{Design of Op-amp 741}

In this section, copy-paste Figure 2 of the lab handout (internal circuit of Op-amp 741) into a picture editing software such as Microsoft Paint, Powerpoint, or Inkscape (very useful open source software for making nice looking illustrations). Annotate the figure by drawing labeled boxes around the parts of the circuit that behave as current mirrors, differential amplifiers, cascode stages, biasing stages. Are there any other stages that you can see apart from these? Try explaining their functionality.

\section{Experimental results}

\subsection{Input offset voltage measurement}

Draw the circuit diagrams of the circuit you used, with the resistor values you used marked clearly. Only diagrams made in xcircuit will be accepted. Copy-pasting from handout will be counted as plagiarism. If you are smart, you will start drawing the most complicated circuit first, and then modify it accordingly for the other diagrams.

Mention your observations. What challenges did you face in performing this part of the experiment?

\subsection{DC open-loop gain measurement}

Draw the circuit diagram of the circuit you used, with the resistor values you used marked clearly. Only diagrams made in xcircuit will be accepted. Copy-pasting from handout will be counted as plagiarism. Did you measure the pot values for null adjustment?

Mention your observations. What challenges did you face in performing this part of the experiment?

\section{Questions for reflection}

1. If the method for null-adjustment is as simple as the one you performed in lab, why isn't the 741 op-amp sold with the offset voltage internally calibrated?\\
Ans. Enter answer here.
\\\\
2. If the temperature in the lab were different from what it was when you performed the experiment, do you expect the pot value you ended up with will still give you offset nullification? Explain your answer. Hint: Look at the internal circuit diagram and figure out what parameters may change when the temperature changes.\\
Ans. Enter answer here.
\\\\
3. What is the slew-rate of an op-amp? Read up the definition and explain it in your own words here. Could you suggest an experiment to measure slew-rate of op-amp 741?\\
Ans. Enter answer here.
\\\\
4. What is the role of capacitor C in the circuit you used in the second part of the lab (i.e. in figure 8 of the hand-out)? (Hint: there is a statement in the hand-out mentioning why C is connected; could you explain why that statement is true?)\\
Ans. Enter answer here.

\end{document}
